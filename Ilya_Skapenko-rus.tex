\documentclass[11pt,a4paper]{moderncv}

\moderncvtheme[orange]{classic}

\usepackage [T2A] {fontenc}   % Кириллица в PDF файле
\usepackage [utf8] {inputenc} % Кодировка текста: utf-8
\usepackage [russian] {babel} % Переносы, лигатуры

\renewcommand{\rmdefault}{cmr} % Шрифт с засечками
\renewcommand{\sfdefault}{cmss} % Шрифт без засечек
\renewcommand{\ttdefault}{cmtt} % Моноширинный шрифт

\usepackage[unicode]{hyperref}
\definecolor{linkcolour}{rgb}{0,0.2,0.6}
\hypersetup{colorlinks,breaklinks,urlcolor=linkcolour, linkcolor=linkcolour}

\firstname{Илья}
\familyname{Скапенко}
\address{}{Ростов-на-Дону, Россия}
\mobile{+7 919 881-82-87}
\email{iliasr@mail.ru}
\extrainfo{Год рождения: 1993}

\begin{document}
\maketitle

\section{Навыки}
  \subsection{Профессиональные}
    \cvline
    {Программист C++}
      {
      Языки: C++, Python, assembler (x86, x86-64)\newline{}
      Библиотеки/фрэймворки: Boost, Qt, Opencv, GTest, STL, WinApi, CUDA \newline{}
      Средства автоматической сборки: CMake, make \newline{}
      Контроль версий: Git, SVN \newline{}
      IDE: MS VS 2017, CLion, Qt Creator, Eclipse (Nsight, CDT)
      }
  \subsection{Языки}
    \cvline
    {Английский}
      {
      Уровень --- upper intermediate \newline{}
      }

\section{Опыт}
\cventry
  {Июнь 2016--Февраль 2017}
  {\bfseries C++ программист}
  {Esignal, Ростов-на-Дону, Россия}
  {\newline{}\url{http://www.esignal.com}}{}
  {Поддержка текущего кода, рефакторинг и добавление новых возможностей.}
\cventry
  {Февраль 2017--н.в}
  {Программист микроконтроллеров}
  {ООО Изоскан, Ростов-на-дону, Россия}
  {\newline{}\url{http://r-call.ru/}}{}
  {Настройка операционных систем, систем сборки, поддержка текущего кода. Разработка программного обеспечения.}

\section{Образование}
  \subsection{Университет}
    \cventry
      {2011--2015 г.}
      {Бакалавриат. Южный Федеральный Университет}
      {Ростов-на-Дону, Россия}
      {Институт математики, механики и компьютерных наук им. И.И. Воровича}
      {Прикладная математика и информатика}
      {}
    \cventry
      {2015--2017 г.}
      {Магистратура. Южный Федеральный Университет}
      {Ростов-на-Дону, Россия}
      {Институт математики, механики и компьютерных наук им. И.И. Воровича}
      {Фундаментальная информатика и информационные технологии}
      {}
  \subsection{Онлайн курсы}
    \cventry
      {Апрель 2013--Июнь 2013}
      {The Hardware/Software Interface}
      {\newline\url{https://www.coursera.org/course/hwswinterface}}
      {}{}{}
    \cventry
      {Июнь 2013--Сентябрь 2013}
      {Computer Networks}
      {\newline\url{https://www.coursera.org/course/comnetworks}}
      {}{}{}
    \cventry
      {Сентябрь 2013--Ноябрь 2013}
      {Quantum Mechanics for Scientists and Engineers}
      {\newline\url{https://lagunita.stanford.edu/courses/Engineering/QMSE01/Quantum_Mechanics_for_Scientists_and_Engineers/info}}
      {}{}{}

\section{Публичная деятельность}
  \subsection{Конференции}
    \cventry
      {Апрель, 2017}
      {Преобразование по уплотнению кода в LLVM}
      {Языки программирования и компиляторы 2017}
      {Ростов-на-Дону, Южный Федеральный Университет}
      {\newline\url{https://elibrary.ru/item.asp?id=29098144}}
      {}{}

\end{document}